\documentclass[12pt]{article}
\headheight=-20pt
\topmargin=0pt
\oddsidemargin=0pt
\textwidth=16cm
\textheight=24cm
\usepackage{latexsym}
\usepackage[T2A]{fontenc}
\usepackage[utf8]{inputenc}
\usepackage[russian]{babel}
\title{\textbf{Применение автоматов для изменения вероятностого распределения алфавита}}
\author{Муравьёв Никита}
\date{}
\begin{document}

\maketitle

\section{Постановка задачи}
Имеется алфавит L. На нем задано вероятностное распределение. Рассматриваются обратимые автоматы, для которых этот алфавит является как входным, так и выходным. Требуется выяснить, каково вероятностное распределение выходного алфавита "в пределе".

\section{Теорема 1}
Групповой автомат\footnote{В каждое состояние входят стрелки всех видов, и из каждого состояния выходят стрелки всех видов}, равномощный алфавиту, у которого совпадают функции перехода и выхода, "в пределе"\ дает равномерное распределение алфавита, если вероятность любого входа больше нуля.

$\triangleright$ Матрица переходов P, соответствующая групповому автомату с n состояниями, будет стохастической с положительными элементами. Для такой матрицы применима  сжимающая лемма, согласно которой $\exists\lim_{t\to \infty} P^{t}=W.$ Причем 
$$W = \left(\begin{array}{ccc}
w_1 & \dots & w_n\\
\vdots & \ddots & \vdots\\
w_1 & \dots & w_n
\end{array}\right)$$
Но матрица P является дважды стохастической\footnote{Остается стохастической после транспонирования.}, так как при ее транспонировании получаем матрицу, соответствующую тому же автомату, но с обращенными стрелками. Значит, W тоже дважды стохастическая. Но тогда $nw_i=1\Rightarrow w_i = \frac{1}{n} \  \forall i.\ \Box$ 

\section{Теорема 2}
Если не групповой автомат равномощен алфавиту, а его функции перехода и выхода совпадают, то он не дает "в пределе" \ равномерного распределения алфавита, если вероятность любого входа больше нуля.

$\triangleright$ Предположим, что $\exists$ автомат A, удовлетворяющий условию, который дает в пределе равномерное распределение, то есть для его матрицы переходов P $$\exists\lim_{t\to\infty}P^t = \frac{1}{n}\left(\begin{array}{ccc}
1 & \dots & 1\\
\vdots & \ddots & \vdots\\
1 & \dots & 1
\end{array}\right)=W$$
Тогда $WP=W$ $\Rightarrow$ $\frac{1}{n}\sum_{i=1}^np_{ij}=\frac{1}{n}\ \Rightarrow\ \sum_{i=1}^np_{ij}=1\ \forall j=1,\dots,n$
\\ Значит, P - дважды стохастическая матрица. Но это значит, что в любое состояние входят стрелки всех видов, ведь элементы матрицы неотрицательны. Из стохастичности матрицы и неотрицательности ее элементов следует, что из любого состояния выходят стрелки всех видов. Выходит, что автомат групповой. Противоречие с условием.\ $\Box$


\end{document}
